\section{Why Digital Predistortion?}
	
	Power amplifiers are used in almost all wireless communication devices. They amplify the communication signal such that a good signal to noise ratio is obtained. They also are an important power consuming block in a communication chain. A power amplifier is often operated in a nonlinear operation mode to improve its efficiency. This nonlinear behavior should be compensated in a later step to reach the strict telecommunication requirements.
	A Digital Pre-Distortion (DPD) is a common technique to linearize the input-output behavior of a power amplifier. With DPD the input signal of the amplifier is modified such that the desired (i.e. linear) behavior is obtained. 

\section{Current Techniques of DPD}
	\subsection{Direct and Indirect Learning}
	\subsection{Nonlinear Models}
\section{The Best Linear Approximation}
	\subsection{What is the BLA?}
	\subsection{How to measure it?}
	\subsection{Out of Band BLA and the Tickler Tone}

\section{Iterative Learning Control}
	\subsection{The Algorithm}
	\subsection{Properties}