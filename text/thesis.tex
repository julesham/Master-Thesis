\documentclass[a4paper]{report}
\usepackage{times}
\usepackage[pdftex]{graphicx}
\usepackage{hyperref}
\usepackage[cmex10]{amsmath}
\usepackage[font=footnotesize]{subfig}
\usepackage{url}

\begin{document}
\title{Design and Realization of a Digital Predistorter for a Power Amplifier}

\author{Jules Hammenecker \\ Brussels Faculty of Engineering \\ Vrije Universiteit Brussel - Universit\'e Libre de Bruxelles}
\date{2014-2015 }

\maketitle
\begin{abstract}
Power amplifiers are used in almost all wireless communication devices. They amplify the communication signal such that a good signal to noise ratio is obtained. They also are an important power consuming block in a communication chain. A power amplifier is often operated in a nonlinear operation mode to improve its efficiency. This nonlinear behavior should be compensated in a later step to reach the strict telecommunication requirements.
PA output spectrum
A Digital Pre-Distortion (DPD) is a common technique to linearize the input-output behavior of a power amplifier. With DPD the input signal of the amplifier is modified such that the desired (linear) behavior is obtained. 
\end{abstract}

\tableofcontents


\begin{thebibliography}
	{09}
	
	\bibitem{100ex} J.~Schoukens, R.~Pintelon, Y.~Rolain , \emph{Mastering System Identification in 100 Exercises.}\hskip 1em plus 0.5em minus 0.4em\relax IEEE Press (2012), 183-238. 

\end{thebibliography}

\end{document}
